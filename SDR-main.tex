\documentclass[xcolor=table,usenames,dvipsnames]{beamer}

\usepackage[british]{babel}
\usepackage{courier}
\usepackage{xcolor}
\usepackage{mdframed}
\usepackage[utf8]{inputenc}
\usepackage{multicol,calc}
\usepackage{ragged2e} 
\usepackage{menukeys}
\usepackage{graphicx}
\usepackage{url}
\usepackage{amsmath, amsthm, amssymb}
\usepackage{setspace} %espaciado
\usepackage{etoolbox} %color bibliografía
\usepackage{wrapfig} %imagen alrededor texto
\usepackage{ajustes}



\title[Laboratorios SDR con GNU Radio (QT-2da Ed)]{
	Laboratorios SDR con GNU Radio}

\author[Ingeniería Electrónica]{Creación Colectiva\\
	\tiny 
	Bryan Estifen Garcia Zamora,
Juan David Cristancho Gaona,
Carlos Fernando Bustos Quevedo,
Carlos Mario Herrera Fernandez,
Esther Alexandra Ramos Arias,
Héctor Javier Vega Lozano,
Ferney Genaro Vasquez Sanabria,
Luis Mateo Cuervo Romero,
Julian Fernando Barreto Coronado,
Sergio Andres Dimas Gomez,
Leonardo Nicolas Solorzano Cruz,
Alex Isaac Urrea Mahecha,
Anderson Trullo Arias,
Pedro Yovan Munca Cadena\\
	\scriptsize
	Tutor\\ 
	\tiny
	Rodríguez Mújica Leonardo}

\institute[Universidad de Cundinamarca]{
	Facultad de Ingeniería\\
	Universidad de Cundinamarca}

\date{Diciembre 2021}

%-----------------------------------

\begin{document}
	
	%espaciado item
	\let\olditemize\itemize
	\def\itemize{\olditemize\itemsep=4pt }
	\footnotesize
	
	\begin{frame}
	\titlepage
\end{frame}
%-----------------------------------

\begin{frame}
\frametitle{Contenidos generales}
\tableofcontents[subsectionstyle=hide/hide, subsubsectionstyle=hide/hide]
\end{frame}
%-----------------------------------


%-----------------------------------

\section{LABORATORIOS CON SOFTWARE}
\begin{frame}

\pgfdeclareimage[width=\paperwidth,height=\paperheight]{bg}{imagenes/fondo_seccion}
\setbeamertemplate{background}{\pgfuseimage{bg}}

\definecolor{greenU}{RGB}{212,202,72}
\setbeamercolor{block body}{fg=Black,bg=greenU}
\begin{block}{}
\centering
\vspace{8mm}
\Large{LABORATORIOS CON SOFTWARE}
\vspace{8mm}
\end{block}
\end{frame}
%-----------------------

{
\begin{frame}
\frametitle{Parte I - Tabla de contenidos}
\begin{spacing}{1.5}
\tableofcontents[currentsection,sectionstyle=hide/hide,subsectionstyle=show/show/hide, subsubsectionstyle=hide]
\end{spacing}
\end{frame}
}

%///////////////////////////////////////////////////////////////
\input{parte1/intro/intro.tex}

%///////////////////////////////////////////////////////////////
\input{parte1/lab1/lab1.tex}




%/////////////////////////

%\input{parte2/SDR-II}

%/////////////////////////

%\input{parte3/SDR-III}

%/////////////////////////

%\input{Modulaciones_digitales/SDR-IV}

%/////////////////////////
%\input{bibliografia/bibliografia.tex}

%/////////////////////////

%\input{soluciones/soluciones}




\end{document}
